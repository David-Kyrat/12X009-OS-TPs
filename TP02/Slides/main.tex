\documentclass{beamer}
\usepackage[utf8]{inputenc}
\newcommand{\num}{02}
\newcommand{\sn}{\secname}
\newcommand{\ssn}{\subsecname}
\beamertemplatenavigationsymbolsempty
\newcommand{\unilogo}[1]{\includegraphics[scale=#1]{images/unige.png}}
%
%
\definecolor{light-gray}{gray}{0.95}
\newcommand{\code}[1]{$\mbox{\colorbox{light-gray}{\texttt{#1}}}$}
\newcommand{\quo}[1]{``{#1}''}
%

\usetheme{Boadilla}
%\useoutertheme{miniframes}

\usecolortheme{whale}

\title[Systèmes d'Exploitation: TP \num]{Systèmes d'Exploitation - Examen}

\subtitle{12X009 - TP\num}
\author[Noah Munz, Sciences Informatiques]{Noah Munz (19-815-489)}
\institute[]{Département d'Informatique \\ Université de Genève}

\date[Examen Oral]{Mardi 31 Janvier 2023}
\logo{\includegraphics[height=1cm]{images/unige.png}}

\AtBeginSection[]{
  \begin{frame}
    \frametitle{Table of Contents}
    \tableofcontents[currentsection]
  \end{frame}
}


\begin{document}

\pdfcompresslevel=0
\pdfobjcompresslevel=0


\frame{\titlepage}

\begin{frame}
\frametitle{Table of Contents}
\tableofcontents
\end{frame}


%
% ------- REAL START -------------
%


%
\section{Rappel: But du TP}
\begin{frame}{\secname \subsecname}
%

\begin{itemize}
    \item Se familiariser avec la manipulation de chaînes de caractères via la manipulation de \code{argv}/\code{argc} ainsi qu'avec le parsing d'option \& paramètre (\code{getopt}, \code{optstring}...) \\
    
    \item Se familiariser avec la liaison de libairies externes (openssl), Makefile...\\
    
    \item Se familiariser avec l'utlisation fonctions de hashages
\end{itemize}

\end{frame}


%
%
\section{TADs \& leurs relations}
\subsection{Décomposition modulaire}
\begin{frame}{\sn : \ssn}
%

TP reste assez simple, seulement 2 modules ont été créés.\\ Le premier \code{OptionParser} (qui, part la suite, sera le début d'un module \code{util} qui sera réutilisé et aggrandi à chaque TP suivant) qui s'occupe de vérifier si l'input de l'utilisateur est valide ou pas, \textit{sépare}, \textit{copie} et \textit{stoque} les différentes parties des différentes entrées en fonctions des options de ces dernières.\\


\vspace{0.5cm}
//INTRO AUTRE MODULE

\end{frame}


\begin{frame}{\sn : \ssn}
    Par exemple \code{OptionParser} contient une fonction \code{checkEnoughArgs()}  (dont le nom est déjà assez explicite), ainsi qu'une autre fonction \code{\scriptsize  int parseArgs(int argc, char* argv[], char** fileToHash[], }\\
    \indent $\quad$ \code{\scriptsize int* fileAmnt, char** stringToHash   )}
    qui va :
    \begin{itemize}
        \item parse les arguments optionnels\\
        \vspace{0.1cm}
        
        \item Si \code{-f} n'as \textit{pas} été fourni $\Rightarrow$ appel une fonction plus simple pour juste hash la concaténation des entrées avec la méthode donnée avec \code{-t}. i.e. va stocker la concaténation dans 
        \code{stringToHash}.\\
        \vspace{0.1cm}
        
        \item Si \code{-f} a été fourni, va extraire chaque nom de fichier et les stocker dans le buffer \code{fileToHash}.
        \vspace{0.1cm}
        
    \end{itemize}

%!! METTRE çA QQ PART POUR LAVOIR A LEXAMEN !!

%/**
 %* Call parseOptArgs to parse given ptions then if "-f" was not provided, 
 %* call parseArgsAsString to interpret all given argument (that are not options) as 1 single string
 %* i.e. "-f file1 file2 [-t <hashMethod>]" parses file1, file2 and hashMethod as separate %things to hash.
 %* and "s1 s2 [-t <hashMethod>]" parses "s1 s2" as 1 string to hash.
 %* 
 %* @param argc Number of arguments passed to the program.
 %* @param argv Array of arguments given to the program.
 %* @param givenFilesToHash pointer to string array into which store the parsed files to hash or null if -f was not provided. Does a malloc. Remember to free! (when according option was chosen)
 %* @param fileAmnt Variable into which store the amount of given files.
 %* @param givenStringToHash pointer to string into which store the parsed string to hash or null if -f was provided. Does a malloc. Remember to free! (when according option was chosen)
 %* @return 0 if success else error code
 %*/
\end{frame}


%
\subsection{Structures de données utilisées}
\begin{frame}{\sn : \ssn}

Pareil qu'avant, l'implémentation de structure n'a pas été nécessaire.
    
\end{frame}
%

%
\subsection{Décomposition fonctionnelle}
\begin{frame}{\sn : \ssn}
    Lorem ipsum dolor sit amet
\end{frame}

%
%
\section{Test realisés pour valider le fonctionnement du TP}
\begin{frame}{\sn}
%

\end{frame}
%
\section{Réponses aux questions (de l'énoncé du TP)}
\begin{frame}{\secname}
%
    Lorem ipsum dolor sit amet

    
\end{frame}

\section{Réponses aux questions (générales)}

\begin{frame}{\secname}
    
\end{frame}

\end{document}

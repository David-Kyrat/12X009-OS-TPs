%
\section{Réponses aux questions (de l'énoncé du TP)}
\begin{frame}{\secname}
%

\begin{itemize}
    \item Combiner la commande echo "Le manuel disait: Nécessite Windows 7 ou mieux. J'ai donc
installé Linux". Avec les commandes ci-dessus pour calculer les hashs sans utiliser de fichier;
le résultat est différent, pourquoi? Comment résoudre le probléme? \\
    \vspace{0.3cm}

    \begin{itemize}
        \item  On voit que lcs 2 hashs different dans les 2 cas car on a un retour la ligne dans le fichier mais pas
        lorsque l'on pipe directement l'output de echo dans l'input dc SHA1 et MD5.
        En effet, on a vu avant qu'en ajoutant ou enlevant un saut de ligne à la fin du fichier \code{string-tohash.txt} on
        pouvant switcher comme on voulait entre ces 2 variantes possibles.
    \end{itemize}
   
\end{itemize}

\end{frame}

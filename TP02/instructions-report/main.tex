\documentclass[draft, french]{article}
\usepackage[margin=1in,a4paper]{geometry}
\usepackage[utf8]{inputenc}
\usepackage[T1]{fontenc}
%\usepackage{hyperref}
\usepackage[fleqn]{amsmath}
\usepackage{enumitem,amssymb}
\usepackage{graphicx}
\usepackage{xcolor}
\usepackage{amsthm}
\usepackage{amsfonts}
\usepackage{pdfpages}
\usepackage{pgfplots}
\usepackage{fancyhdr}
\usepackage{pdfpages}
\usepackage{lastpage}
\usepackage{cleveref}
\usepackage{ragged2e}
\usepackage{graphicx}
%
\pagestyle{fancy}
\pgfplotsset{compat=newest}
\usetikzlibrary{calc}
\setlength{\headheight}{73.96703pt}
\addtolength{\topmargin}{-49.96703pt}
\setlength{\footskip}{45.81593pt}
\setlength{\parindent}{0cm}
%
\hypersetup{colorlinks=true,
    linkcolor=blue,
    urlcolor=blue,
}
\urlstyle{same}
%
\newcommand{\unilogo}[1]{\includegraphics[scale=#1]{images/unige.png}}
%
%
\definecolor{light-gray}{gray}{0.95}
\newcommand{\code}[1]{$\mbox{\colorbox{light-gray}{\texttt{#1}}}$}
\newcommand{\quo}[1]{``{#1}''}
%
\newcommand{\R}{\mathbb{R}}
\newcommand{\N}{\mathbb{N}}
\newcommand{\es}{\varnothing}
\DeclareMathOperator{\Ima}{Im}
\renewcommand{\and}{\wedge}
\newcommand{\ra}{\rightarrow}
\newcommand{\inv}[1]{{#1}^{-1}}
\newcommand{\br}[1]{\{{#1}\}}
\newcommand{\fa}{\forall}
\newcommand{\te}{\exists}
\newcommand{\dom}[1]{\mathcal{D}_{#1}}
%
\newcommand{\set}[2]{\{{#1}\ |\ {#2}\}}
\newcommand{\w}{\omega}
\newcommand{\s}{\Sigma}
%
\newcommand{\xor}{\oplus}
\newcommand{\nb}{02}
%
%
\title{
   {\huge Université de Genève \\ - \\ Sciences Informatiques} \\
    \vspace{0.6cm}
    \unilogo{0.41} \\ 
    \vspace{1.1cm}
    {\huge Systèmes d'Exploitation - TP \nb}
    \vspace{0.1cm}
}
\author{Noah Munz - Gregory Sedykh}
\date{Octobre 2022}

%
\lhead{Université de Genève \\ Sciences Informatiques}
\rhead{Noah Munz - Gregory Sedykh \\ Systèmes d'Exploitation - TP\nb}
\cfoot{} \lfoot{\hspace{-1.8cm} \unilogo{0.06}}
\rfoot{Page \thepage \hspace{0.5mm} / \pageref{LastPage} \hspace{-1cm}}
%
%
\begin{document}
\pdfcompresslevel=0
\pdfobjcompresslevel=0
%
\maketitle
\thispagestyle{empty}
\clearpage
\setcounter{page}{1}
%
%
\begin{center}
{\huge TP \nb}
\end{center}
\vspace{0.2cm}
%
\stepcounter{section}
\section{Cryptographie}
%
\vspace{0.3cm}
\subsection{Concepts}
Exercice: Hasher le string \quo{Le manuel disait: Nécessite Windows 7 ou mieux. J'ai donc installé Linux} avec SHA1 puis MD5.
La première fois en lui passant un fichier contenant la string, la deuxième fois en combinant echo avec les SHA1 et MD5.\\


SHA1 dans fichier: 74bac9f1611450f44e0a377da1b0732034d20f59\\
MD5 dans fichier: ef81ae80ec507096f761cefa49f7b63e
\\
\\SHA1 avec echo: aa91cf14c329a5fe9b1158ae2665b39370f57e37
\\MD5 avec echo: 120227d6118cfddbd21639644aa0884d\\

On voit que les 2 hashs different dans les 2 cas car on a un retour à la ligne dans le fichier mais pas lorsque l'on pipe directement l'output de \code{echo} dans l'input de SHA1 et MD5.

%
\vspace{0.3cm}
\subsection{La libraire OpenSSL}

La fonction de l'exemple man est dans le fichier \code{examples/openssl\_example.c}.

On voit bien que en changeant le message à \quo{Le manuel disait: Nécessite Windows 7 ou mieux. J'ai donc installé Linux}, on recoit le même hash que dans la partie (2.1).
%

%
%
\vspace{0.3cm}
\section{Gestion des paramètres d'un programme}
\vspace{0.3cm}

La fonction de l'exemple man est dans le fichier \code{getopt\_example.c}

\vspace{0.3cm}
\section{Intégration: Le programme à réaliser}
\subsection{Contenu du projet}

\subsubsection*{(1) Code}

Le code de l'implémentation, qui se compile avec la
\textit{\quo{target}} \code{all} du \code{Makefile}, se trouve dans:
\begin{itemize}
    \item Le fichier \code{main.c} qui contient la fonction main qui permet de lancer le programme.\\
    Son fonctionnement sera expliqué en détail ci-après
    (l'executable qui lui fait directement appel est compilé, à la racine, sous le nom \code{digest}.)
    
    \item Le module \code{OptionParser} qui définit les différentes options disponibles et contient tout le\\ nécessaire au traitement des options et arguments spécifiés par l'utilisateur.
    
    \item Le module \code{hash\_calc} qui s'occupe de calculer les hashs de tout ce qui a été demandé. 
\end{itemize}

\newpage

\subsubsection*{(2) Reste}
Le reste du projet contient :

\begin{itemize}
    \item Un dossier \code{exemples} qui contient l'implémentation demandée des examples du manuel. Ils peuvent être compilés avec les targets \code{<nomDuTest>\_example} du Makefile.
    
    \item Un dossier \code{res} qui contient des fichiers à être utilisé pour tester le calcul des hashs. 
    
    \item Un dossier \code{out} qui contient le code compilé et les \code{.o} des différents modules.
    
    \item Et finalement, le Makefile.
    
\end{itemize}

\subsection{Manuel}

La commande \code{make all} permet de compiler tous les fichiers.\\

Comme dit précédemment, le makefile donne aussi d'autres possibilités de compilation (i.e. compiler seulement les \code{.o} situé dans \code{out/}, les exemples dans \code{exemples/} \ldot) \\

On peut exécuter le programme de 4 manières différentes:


\begin{enumerate}
    \item[\textbf{1-2)}] En spécifiant un ou plusieurs fichiers avec \code{-f}, en spécifiant (ou pas) la fonction de hashage \code{-t} :\\ 
    \code{./digest -f file1 file2 \ldot [-t <hashMethod>]}.\\
    
    Ce qui retournera \textbf{1} hash \textbf{par} fichier. E.g.:\\
    \code{\$ ./digest -f res/test.txt res/string-tohash.txt -t md5}\\
    
    \code{Hashing Method: md5}
    
    \code{Hashing file "res/string-tohash.txt"\ldot}\\
    \code{120227d6118cfddbd21639644aa0884d File: res/string-tohash.txt}\\ (ou \code{ef81ae80ec507096f761cefa49f7b63e} sans \quo{final newline})
    \code{------------------------}\\
    
    \code{Hashing file "res/test.txt"\ldot}\\
    \code{4e1243bd22c66e76c2ba9eddc1f91394e57f9f83 File: res/test.txt}\\
    \code{------------------------}\\
    
    \item[\textbf{3-4})] En spécifiant une ou plusieures chaînes de caractères, en spécifiant (ou pas) la fonction de hashage \code{-t} :\\
    \code{./digest string1 string2 \ldot [-t <hashMethod>]}.\\
    
     Ce qui retournera \textbf{1} hash \textbf{pour} la concaténation de \textbf{tous} les strings. E.g.:\\
    
    \code{\$ ./digest -t sha256 max linux, windows = linux}\\
    
    \code{Hashing Method: sha256}
    
    \code{String to hash: "max linux, windows = linux"}\\
    \code{Digest: 069da09aa2f501f87b1603db31659261bddd89adae1bd6e895b3d36bbb495250}\\
    
\end{enumerate}

(l'ordre de l'option \code{-t} n'est pas important, elle peut être au début ou à la fin.)\\

Pour vérifier, \code{sha1sum string-tohash.txt} et \code{./digest -f res/string-tohash.txt} donnent le même hash digest (voir 2.1)


\end{document}

\documentclass{beamer}
\usepackage[utf8]{inputenc}
\newcommand{\num}{03}
\newcommand{\sn}{\secname}
\newcommand{\ssn}{\subsecname}
\beamertemplatenavigationsymbolsempty

\usetheme{Boadilla}
%\useoutertheme{miniframes}

\usecolortheme{whale}

\title[Systèmes d'Exploitation: TP \num]{Systèmes d'Exploitation - Examen}

\subtitle{12X009 - TP\num}
\author[Noah Munz, Sciences Informatiques]{Noah Munz (19-815-489)}
\institute[]{Département d'Informatique \\ Université de Genève}

\date[Examen Oral]{Mardi 31 Janvier 2023}
\logo{\includegraphics[height=1cm]{images/unige.png}}

\AtBeginSection[]{
  \begin{frame}
    \frametitle{Table of Contents}
    \tableofcontents[currentsection]
  \end{frame}
}
\begin{document}

\frame{\titlepage}

\begin{frame}
\frametitle{Table of Contents}
\tableofcontents
\end{frame}


%
% ------- REAL START -------------
%


%
\section{Rappel: But du TP}
\begin{frame}{\secname \subsecname}
%

This is a text in second frame. For the sake of showing an example.

\end{frame}


%
%
\section{TADs \& leurs relations}
\subsection{Décomposition modulaire}
\begin{frame}{\sn : \ssn}
%

Lorem ipsum dolor sit amet

\end{frame}

%
\subsection{Structures de données utilisées}
\begin{frame}{\sn : \ssn}

Lorem ipsum dolor sit amet
    
\end{frame}
%

%
\subsection{Décomposition fonctionnelle}
\begin{frame}{\sn : \ssn}
    Lorem ipsum dolor sit amet
\end{frame}

%
%
\section{Test realisés pour valider le fonctionnement du TP}
\begin{frame}{\sn}
%

\end{frame}
%
\section{Réponses aux questions (de l'énoncé du TP)}
\begin{frame}{\secname}
%
    Lorem ipsum dolor sit amet

    
\end{frame}

\section{Réponses aux questions (générales)}

\begin{frame}{\secname}
    
\end{frame}

\end{document}
%
\section{Réponses aux questions (de l'énoncé du TP)}
\begin{frame}{\secname}

\vspace{-0.7cm}
\begin{enumerate}
    {\footnotesize 
    \item  \emph{Que se passera-t-il si nous déverrouillons un fichier (ou une partie du fichier) qui n’est pas verrouillé?}
    \begin{itemize}
       { \footnotesize 
       \item  On reçoit erreur \code{EINVAL}
        et \code{strerror(errno)} donne\\
        \quo{\textit{Invalid Argument}}  (Source: \code{man 2 flock})}
        
    \end{itemize}

\vspace{0.2cm}
   \item \emph{Que se passera-t-il si nous mettons un nouveau verrou sur une section déjà verrouillée? Le type de verrou changera t-il le résultat? Expliquer dans la situation avec le même processus et avec 2 processus différents.}\\
   
    \begin{itemize}
    { \footnotesize 
        \item  Lorsqu'on a 2 processus différents,
  Le type de verrou changera le résultat: si on a déjà un read lock sur un fichier, on peut quand même mettre des read locks dessus. Par contre, si un fichier a un write lock, on ne pourra mettre ni de read lock, ni de write lock supplémentaires.\\
  Pareil si on essai de mettre un write lock sur une portion de fichier protégé par un read lock.
  ont obtient l'erreur \code{Resource temporarily unavailable}. \textit{(Voir screenshot slide \textbf{12})}\\   
  
  Par contre si on est dans le cas d'un seul processus, il n'y a pas ce problème. En effet, il ne peut s'auto-empêcher d'accéder / vérouiller le contenu qu'il a lui même protégé.}
    \end{itemize}}
   
\end{enumerate}

\end{frame}

\section{TADs \& leurs relations}
\subsection{Décomposition modulaire}

\bframes{
\small
3 modules (+ \texttt{server.c}, \texttt{client.c}) ont été créés pour la réalisation du TP\@:
\begin{enumerate}
     \item Le module \code{optprsr.c} (repris des TPs 2 à 4) \\
    contient 5 fonctions, servant à l'extraction \& copie\ldots des arguments passés paramètres.\\
    \vspace{0.3cm}

    \item Le module \code{util.c} (repris des TPs 3 à 4) \\
      Fonctions servant divers usages allant de la gestion d'erreurs à la gestion de chaînes de caractères, en passant par des wrappers qui incluent lesdites fonctions de gestions d'erreurs.\\
    
\end{enumerate}
}


\bframes{
\begin{enumerate}
    \stepcounter{enumi} \stepcounter{enumi}

    \item Le module \code{functions} qui gères les travaux communs que doivent faire le client et le serveur pour pouvoir communiquer ensemble (e.g. création d'une struct \code{sockaddr\_in})

    \vspace{0.3cm}

    \item \code{client.c} Contient le \quo{driver code} pour se connecter au server et jouer avec lui.
    Se décompose en une partie \textit{connection}, (i.e. création de socket, connection au socket)\\ 
    puis en une partie \textit{jeu} (i.e. lecture/écriture sur la socket)
    
    \vspace{0.3cm}

    \item \code{server.c} Contient le \quo{driver code} pour ouvrir des sockets attendre des connections clients et créé des enfants pour jouer avec les clients qui arrivent.
    Se décompose en une partie \textit{connection}, (i.e. création de socket, attente \& acceptations de nouveau clients) géré par le parent\\ 
    puis en une partie \textit{jeu} (i.e. lecture/écriture sur la socket) géré par les enfants.
            
    \end{enumerate}
}

%

\subsection{Structures de données utilisées}
\bframes{
    L'implémentation de structure n'a pas été nécessaire.
}

%
%
\subsection{Décomposition fonctionnelle}
\bframes{
    \textbf{Client}:\\
    \vspace{0.3cm}
     Pour faire tout ce dont il a besoin, le client doit:
     \begin{enumerate}
         \item Créer une socket 
         \item Se connecter au server en initiant la connection sur la socket avec l'adresse et le port donné
         \item Lire/écrire de/sur la socket
         \item Fermer la socket
     \end{enumerate}
}

\bframes{
     Pour ceci, nous avons implémenté les différentes fonctions suivantes:
     \begin{enumerate}
         \item module \code{functions} : \code{int new\_socket()} crée un nouveau socket de famille \code{AF\_INET}, type \code{SOCK\_STREAM} et de protocole 0 (TCP/IP)
         \item même module : \code{sockaddr\_in new\_sockaddr(int port, const char* addr\_reppr)} \\
         Créé une socket address pour un client ou un server où \code{port} est le numéro de port à partir duquel on va \quo{bind} la socket
         et \code{addr\_repr} est un string (qui peut être nul quand on l'appel depuis \code{server}) qui est simplement la représentation 
         de l'addresse IP (par défaut vaut \code{INADDR\_ANY}) \\
         
         retourne une \code{struct sockaddr\_in}, il ne reste plus qu'à appeler \code{socket()} pour initier la connection (client) et finir l'étape binding-listen-accept (server)
     \end{enumerate}
}

\bframes{
    Pour la lecture/écriture on a simplement utilisé les appels systèmes \code{read()} et \code{write()}. 
    En effet, les sockets sont encapsulés par des descripteur de fichiers i.e. tous les appels qui opèrent sur des socket prennent un descripteur de fichier en paramètre comme si c'était un fichier normal et font les opérations nécéssaire avec.
    
    \vspace{0.3cm}
    
    De la même manière, pour fermer la socket on a juste à utiliser \code{close()}.
}


\bframes{
\textbf{Serveur:}\\
\vspace{0.3cm}
     Pour faire tout ce dont il a besoin, le serveur doit:
     \begin{enumerate}
         \item Créer une socket 
         \item \textit{\quo{bind}} la socket serveur à une adresse
         \item Ecouter/attendre le début d'une connection client
         \item Accepter la connection, obtenir la socket client et créer un enfant pour s'en occuper
         pour pouvoir continuer à attendre d'autre connections en même temps 
         \item Lire/écrire de/sur la socket
         \item Répéter l'étape 4 dès que nécessaire 
         \item fermer la socket client
         \item fermet la socket serveur
     \end{enumerate}
}

\bframes{
\vspace{-0.1cm}
Pour ceci, nous avons implémenté les différentes fonctions suivantes:
Les points 1) 5) 7) et 8) ont déjà été abordés
\begin{enumerate}
    \stepcounter{enumi} 
    \item On a simplement appelé la fonction \code{bind()} avec une \code{struct sockaddr\_in} et un file descriptor lié à la socket, crée comme vu plus haut.
    
    \item Pour la partie écouter et gérer les clients en même temps on a une boucle infine du type 
    \code{for (;;) \{...\}} où au début de la boucle notre processus attend avec un appel à \code{accept()} 
    puis dès qu'il a fini d'attendre (i.e. on est juste à la ligne d'après) il va \code{fork()}
    et laisser l'enfant s'occuper et continue sa boucle.\\
    
    \vspace{0.2cm}
    
    {\footnotesize En réalité pour ce tp on fait un double fork pour faire en sorte que tous les procesus qui discutent directement avec le client soit des petits-enfants du processus qui attend les connections et que leurs parent direct meurent. (afin qu'ils deviennent tous orphelins et qu'ils soit récupéré par \code{init} et terminés correctment à leurs mort, pour éviter les zombies)
    En effet à cet moment nous n'avions pas encore vu la gestions de signaux, avec \code{SIGCHLD}\ldots }
    
    
\end{enumerate}
}

\documentclass[french]{article}
\usepackage[margin=1in,a4paper]{geometry}
\usepackage[utf8]{inputenc}
\usepackage[T1]{fontenc}
\usepackage[autolanguage]{numprint}
\usepackage{hyperref}
\usepackage[fleqn]{amsmath}
\usepackage{enumitem,amssymb}
\usepackage{graphicx}
\usepackage{xcolor}
\usepackage{amsthm}
\usepackage{amsfonts}
\usepackage{pdfpages}
\usepackage{pgfplots}
\usepackage{fancyhdr}
\usepackage{pdfpages}
\usepackage{lastpage}
\usepackage{cleveref}
\usepackage{ragged2e}
\usepackage{graphicx}
%
\pagestyle{fancy}
\pgfplotsset{compat=newest}
\usetikzlibrary{calc}
\setlength{\headheight}{73.96703pt}
\addtolength{\topmargin}{-49.96703pt}
\setlength{\footskip}{45.81593pt}
\setlength{\parindent}{0cm}
%
\hypersetup{colorlinks=true,
    linkcolor=blue,
    urlcolor=blue,
}
\urlstyle{same}
%
\newcommand{\unilogo}[1]{\includegraphics[scale=#1]{images/unige.png}}
%
%
\definecolor{light-gray}{gray}{0.95}
\newcommand{\code}[1]{$\mbox{\colorbox{light-gray}{\texttt{#1}}}$}
\newcommand{\quo}[1]{``{#1}''}
%
\newcommand{\R}{\mathbb{R}}
\newcommand{\N}{\mathbb{N}}
\newcommand{\es}{\varnothing}
\DeclareMathOperator{\Ima}{Im}
\renewcommand{\and}{\wedge}
\newcommand{\ra}{\rightarrow}
\newcommand{\inv}[1]{{#1}^{-1}}
\newcommand{\br}[1]{\{{#1}\}}
\newcommand{\fa}{\forall}
\newcommand{\te}{\exists}
\newcommand{\dom}[1]{\mathcal{D}_{#1}}
%
\newcommand{\set}[2]{\{{#1}\ |\ {#2}\}}
\newcommand{\w}{\omega}
\newcommand{\s}{\Sigma}
%
\newcommand{\xor}{\oplus}
\newcommand{\nb}{06}
%
%
\title{\vspace{-2.5cm}
   {\huge Université de Genève \\ - \\ Sciences Informatiques} \\
    \vspace{0.6cm}
    \unilogo{0.38} \\
    \vspace{1.1cm}
    {\huge Systèmes d'Exploitation - TP \nb}
    \vspace{0.1cm}
}
\author{Noah Munz - Gregory Sedykh}
\date{Decembre 2022}

%
\lhead{Université de Genève \\ Sciences Informatiques}
\rhead{Noah Munz - Gregory Sedykh \\ Systèmes d'Exploitation - TP\nb}
\cfoot{} \lfoot{\hspace{-1.8cm} \unilogo{0.06}}
\rfoot{Page \thepage \hspace{0.5mm} / \pageref{LastPage} \hspace{-1cm}}
%
%
\begin{document}
%
\maketitle
\vspace{0.3cm}
\thispagestyle{empty}
\clearpage
\setcounter{page}{1}
%
%
\begin{center}
{\huge TP \nb}
\end{center}
\vspace{0.3cm}

\section*{Manuel du programme}

\subsection*{Lancement du programme}
Pour lancer le shell, il suffit d'executer la commande \code{./shell}.

\subsection*{Commandes built-in}
\begin{itemize}
    \item \code{cd} : permet de changer de répertoire courant.
    \item \code{pwd} : permet d'afficher le répertoire courant.
    \item \code{exit} : permet de quitter le shell.
\end{itemize}

\subsection*{Jobs}
Les jobs (autres commandes) sont évaluées par le shell, il suffit de l'executer comme dans un shell ordinaire.

\subsection*{Background Jobs}
Pour lancer un job en background, il suffit de mettre un \code{\&} à la fin de la commande.

% \subsection*{Changement de couleur du shell}
% Pour changer la couleur du shell, il suffit de lancer la commande 

\subsection*{Shortcuts}
\begin{itemize}
    % \item \code{Ctrl + D} : permet de quitter le shell.
    \item \code{Ctrl + C} : permet d'annuler la commande en cours.
    \item \code{Ctrl + D} : permet de fermer le shell.
\end{itemize}

\subsection*{Autres}
\begin{itemize}
    \item On a utilisé la librairie \code{readline.h} pour la lecture des commandes, qu'on a eu le droit d'utiliser.
    
    Une fonction implementé par nous-même appelée \code{readInput()} est similaire, mais \code{readline} le fait mieux que nous puisqu'elle contient plus de fonctionnalités (par exemple: avancer et reculer dans la commande, etc.)

    Pour tester le shell sans la librairie \code{readline.h}, il suffit de commenter les \code{\#include} concernant la librairie \code{readline} et de décommenter le code de \code{readInput()} dans \code{input.c}.\\

    Readline peut être installé avec les 2 commandes ensemble:\\
    \code{sudo apt-get install lib32readline8 lib32readline-dev}\\
    \code{sudo apt-get install libreadline-dev}
    \\
    \item \code{Ctrl + D} ne marchera pas si on a un foreground job.

\end{itemize}




\end{document}

\documentclass[french]{article}
\usepackage[margin=1in,a4paper]{geometry}
\usepackage[utf8]{inputenc}
\usepackage[T1]{fontenc}
\usepackage[autolanguage]{numprint}
\usepackage{hyperref}
\usepackage[fleqn]{amsmath}
\usepackage{enumitem,amssymb}
\usepackage{graphicx}
\usepackage{xcolor}
\usepackage{amsthm}
\usepackage{amsfonts}
\usepackage{pdfpages}
\usepackage{pgfplots}
\usepackage{fancyhdr}
\usepackage{pdfpages}
\usepackage{lastpage}
\usepackage{cleveref}
\usepackage{ragged2e}
\usepackage{graphicx}
%
\pagestyle{fancy}
\pgfplotsset{compat=newest}
\usetikzlibrary{calc}
\setlength{\headheight}{73.96703pt}
\addtolength{\topmargin}{-49.96703pt}
\setlength{\footskip}{45.81593pt}
\setlength{\parindent}{0cm}
%
\hypersetup{colorlinks=true,
    linkcolor=blue,
    urlcolor=blue,
}
\urlstyle{same}
%
\newcommand{\unilogo}[1]{\includegraphics[scale=#1]{images/unige.png}}
%
%
\definecolor{light-gray}{gray}{0.95}
\newcommand{\code}[1]{$\mbox{\colorbox{light-gray}{\texttt{#1}}}$}
\newcommand{\quo}[1]{``{#1}''}
%
\newcommand{\R}{\mathbb{R}}
\newcommand{\N}{\mathbb{N}}
\newcommand{\es}{\varnothing}
\DeclareMathOperator{\Ima}{Im}
\renewcommand{\and}{\wedge}
\newcommand{\ra}{\rightarrow}
\newcommand{\inv}[1]{{#1}^{-1}}
\newcommand{\br}[1]{\{{#1}\}}
\newcommand{\fa}{\forall}
\newcommand{\te}{\exists}
\newcommand{\dom}[1]{\mathcal{D}_{#1}}
%
\newcommand{\set}[2]{\{{#1}\ |\ {#2}\}}
\newcommand{\w}{\omega}
\newcommand{\s}{\Sigma}
%
\newcommand{\xor}{\oplus}
\newcommand{\nb}{07}
%
%
\title{\vspace{-2.5cm}
   {\huge Université de Genève \\ - \\ Sciences Informatiques} \\
    \vspace{0.6cm}
    \unilogo{0.38} \\
    \vspace{1.1cm}
    {\huge Systèmes d'Exploitation - TP \nb}
    \vspace{0.1cm}
}
\author{Noah Munz - Gregory Sedykh}
\date{Decembre 2022}

%
\lhead{Université de Genève \\ Sciences Informatiques}
\rhead{Noah Munz - Gregory Sedykh \\ Systèmes d'Exploitation - TP\nb}
\cfoot{} \lfoot{\hspace{-1.8cm} \unilogo{0.06}}
\rfoot{Page \thepage \hspace{0.5mm} / \pageref{LastPage} \hspace{-1cm}}
%
%
\begin{document}
%
\maketitle
\vspace{0.3cm}
\thispagestyle{empty}
\clearpage
\setcounter{page}{1}
%
%
\begin{center}
{\huge TP \nb}
\end{center}
\vspace{0.3cm}

\section*{Description fonctionnelle}
Dans \code{cuisinier.c}, il y a:\\
D'abord une déclaration des sémaphores.\\
Ensuite, il y 4 fonctions:
\begin{itemize}
    \item \code{init\_semaphores}: Crée et initialise les sémaphores.
    \item \code{destroy\_semaphores}: Ferme et unlink les sémaphores.
    \item \code{cook\_pizza}: Va "faire" une pizza entre 1 et 5 secondes, va mettre à jour le sémaphore shelf et va informer le sémaphore serveur qu'il y a une pizza disponible. Si le sémaphore shelf est à 3, il va attendre.
    \item \code{main}: Va créer une mémoire partagée, attendre que le serveur soit prêt, puis initalise les sémpahores, "fait" 10 pizzas, puis détruit les sémaphores ainsi que la mémoire partagée.
\end{itemize}


Dans \code{serveur.c}, il y a:\\
D'abord une déclaration des sémaphores.\\
Ensuite, il y 4 fonctions:
\begin{itemize}
    \item \code{init\_semaphores}: Initialise les sémaphores (sans les créer, car le cuisinier les a déjà crée).
    \item \code{destroy\_semaphores}: Ferme et unlink les sémaphores.
    \item \code{serve\_pizza}: Va "servir" une pizza entre 1 et 5 secondes et va mettre à jour le sémaphore shelf. Si le sémaphore shelf n'est plus à 3, il va informer le cuisinier.
    \item \code{main}: Va ouvrir la mémoire partagée créée par le cuisnier et va se mettre prêt. Ensuite, il initalise les sémpahores, "sert" 10 pizzas, puis détruit les sémaphores ainsi que la mémoire partagée.
\end{itemize}


\section*{Manuel du programme}
\code{cuisinier} peut être exécuté avec la commande suivante: \code{./cuisinier}\\
\code{serveur} peut être exécuté avec la commande suivante: \code{./serveur}

\end{document}

